
% Default to the notebook output style

    


% Inherit from the specified cell style.




    
\documentclass[11pt]{article}

    
    
    \usepackage[T1]{fontenc}
    % Nicer default font (+ math font) than Computer Modern for most use cases
    \usepackage{mathpazo}

    % Basic figure setup, for now with no caption control since it's done
    % automatically by Pandoc (which extracts ![](path) syntax from Markdown).
    \usepackage{graphicx}
    % We will generate all images so they have a width \maxwidth. This means
    % that they will get their normal width if they fit onto the page, but
    % are scaled down if they would overflow the margins.
    \makeatletter
    \def\maxwidth{\ifdim\Gin@nat@width>\linewidth\linewidth
    \else\Gin@nat@width\fi}
    \makeatother
    \let\Oldincludegraphics\includegraphics
    % Set max figure width to be 80% of text width, for now hardcoded.
    \renewcommand{\includegraphics}[1]{\Oldincludegraphics[width=.8\maxwidth]{#1}}
    % Ensure that by default, figures have no caption (until we provide a
    % proper Figure object with a Caption API and a way to capture that
    % in the conversion process - todo).
    \usepackage{caption}
    \DeclareCaptionLabelFormat{nolabel}{}
    \captionsetup{labelformat=nolabel}

    \usepackage{adjustbox} % Used to constrain images to a maximum size 
    \usepackage{xcolor} % Allow colors to be defined
    \usepackage{enumerate} % Needed for markdown enumerations to work
    \usepackage{geometry} % Used to adjust the document margins
    \usepackage{amsmath} % Equations
    \usepackage{amssymb} % Equations
    \usepackage{textcomp} % defines textquotesingle
    % Hack from http://tex.stackexchange.com/a/47451/13684:
    \AtBeginDocument{%
        \def\PYZsq{\textquotesingle}% Upright quotes in Pygmentized code
    }
    \usepackage{upquote} % Upright quotes for verbatim code
    \usepackage{eurosym} % defines \euro
    \usepackage[mathletters]{ucs} % Extended unicode (utf-8) support
    \usepackage[utf8x]{inputenc} % Allow utf-8 characters in the tex document
    \usepackage{fancyvrb} % verbatim replacement that allows latex
    \usepackage{grffile} % extends the file name processing of package graphics 
                         % to support a larger range 
    % The hyperref package gives us a pdf with properly built
    % internal navigation ('pdf bookmarks' for the table of contents,
    % internal cross-reference links, web links for URLs, etc.)
    \usepackage{hyperref}
    \usepackage{longtable} % longtable support required by pandoc >1.10
    \usepackage{booktabs}  % table support for pandoc > 1.12.2
    \usepackage[inline]{enumitem} % IRkernel/repr support (it uses the enumerate* environment)
    \usepackage[normalem]{ulem} % ulem is needed to support strikethroughs (\sout)
                                % normalem makes italics be italics, not underlines
    

    
    
    % Colors for the hyperref package
    \definecolor{urlcolor}{rgb}{0,.145,.698}
    \definecolor{linkcolor}{rgb}{.71,0.21,0.01}
    \definecolor{citecolor}{rgb}{.12,.54,.11}

    % ANSI colors
    \definecolor{ansi-black}{HTML}{3E424D}
    \definecolor{ansi-black-intense}{HTML}{282C36}
    \definecolor{ansi-red}{HTML}{E75C58}
    \definecolor{ansi-red-intense}{HTML}{B22B31}
    \definecolor{ansi-green}{HTML}{00A250}
    \definecolor{ansi-green-intense}{HTML}{007427}
    \definecolor{ansi-yellow}{HTML}{DDB62B}
    \definecolor{ansi-yellow-intense}{HTML}{B27D12}
    \definecolor{ansi-blue}{HTML}{208FFB}
    \definecolor{ansi-blue-intense}{HTML}{0065CA}
    \definecolor{ansi-magenta}{HTML}{D160C4}
    \definecolor{ansi-magenta-intense}{HTML}{A03196}
    \definecolor{ansi-cyan}{HTML}{60C6C8}
    \definecolor{ansi-cyan-intense}{HTML}{258F8F}
    \definecolor{ansi-white}{HTML}{C5C1B4}
    \definecolor{ansi-white-intense}{HTML}{A1A6B2}

    % commands and environments needed by pandoc snippets
    % extracted from the output of `pandoc -s`
    \providecommand{\tightlist}{%
      \setlength{\itemsep}{0pt}\setlength{\parskip}{0pt}}
    \DefineVerbatimEnvironment{Highlighting}{Verbatim}{commandchars=\\\{\}}
    % Add ',fontsize=\small' for more characters per line
    \newenvironment{Shaded}{}{}
    \newcommand{\KeywordTok}[1]{\textcolor[rgb]{0.00,0.44,0.13}{\textbf{{#1}}}}
    \newcommand{\DataTypeTok}[1]{\textcolor[rgb]{0.56,0.13,0.00}{{#1}}}
    \newcommand{\DecValTok}[1]{\textcolor[rgb]{0.25,0.63,0.44}{{#1}}}
    \newcommand{\BaseNTok}[1]{\textcolor[rgb]{0.25,0.63,0.44}{{#1}}}
    \newcommand{\FloatTok}[1]{\textcolor[rgb]{0.25,0.63,0.44}{{#1}}}
    \newcommand{\CharTok}[1]{\textcolor[rgb]{0.25,0.44,0.63}{{#1}}}
    \newcommand{\StringTok}[1]{\textcolor[rgb]{0.25,0.44,0.63}{{#1}}}
    \newcommand{\CommentTok}[1]{\textcolor[rgb]{0.38,0.63,0.69}{\textit{{#1}}}}
    \newcommand{\OtherTok}[1]{\textcolor[rgb]{0.00,0.44,0.13}{{#1}}}
    \newcommand{\AlertTok}[1]{\textcolor[rgb]{1.00,0.00,0.00}{\textbf{{#1}}}}
    \newcommand{\FunctionTok}[1]{\textcolor[rgb]{0.02,0.16,0.49}{{#1}}}
    \newcommand{\RegionMarkerTok}[1]{{#1}}
    \newcommand{\ErrorTok}[1]{\textcolor[rgb]{1.00,0.00,0.00}{\textbf{{#1}}}}
    \newcommand{\NormalTok}[1]{{#1}}
    
    % Additional commands for more recent versions of Pandoc
    \newcommand{\ConstantTok}[1]{\textcolor[rgb]{0.53,0.00,0.00}{{#1}}}
    \newcommand{\SpecialCharTok}[1]{\textcolor[rgb]{0.25,0.44,0.63}{{#1}}}
    \newcommand{\VerbatimStringTok}[1]{\textcolor[rgb]{0.25,0.44,0.63}{{#1}}}
    \newcommand{\SpecialStringTok}[1]{\textcolor[rgb]{0.73,0.40,0.53}{{#1}}}
    \newcommand{\ImportTok}[1]{{#1}}
    \newcommand{\DocumentationTok}[1]{\textcolor[rgb]{0.73,0.13,0.13}{\textit{{#1}}}}
    \newcommand{\AnnotationTok}[1]{\textcolor[rgb]{0.38,0.63,0.69}{\textbf{\textit{{#1}}}}}
    \newcommand{\CommentVarTok}[1]{\textcolor[rgb]{0.38,0.63,0.69}{\textbf{\textit{{#1}}}}}
    \newcommand{\VariableTok}[1]{\textcolor[rgb]{0.10,0.09,0.49}{{#1}}}
    \newcommand{\ControlFlowTok}[1]{\textcolor[rgb]{0.00,0.44,0.13}{\textbf{{#1}}}}
    \newcommand{\OperatorTok}[1]{\textcolor[rgb]{0.40,0.40,0.40}{{#1}}}
    \newcommand{\BuiltInTok}[1]{{#1}}
    \newcommand{\ExtensionTok}[1]{{#1}}
    \newcommand{\PreprocessorTok}[1]{\textcolor[rgb]{0.74,0.48,0.00}{{#1}}}
    \newcommand{\AttributeTok}[1]{\textcolor[rgb]{0.49,0.56,0.16}{{#1}}}
    \newcommand{\InformationTok}[1]{\textcolor[rgb]{0.38,0.63,0.69}{\textbf{\textit{{#1}}}}}
    \newcommand{\WarningTok}[1]{\textcolor[rgb]{0.38,0.63,0.69}{\textbf{\textit{{#1}}}}}
    
    
    % Define a nice break command that doesn't care if a line doesn't already
    % exist.
    \def\br{\hspace*{\fill} \\* }
    % Math Jax compatability definitions
    \def\gt{>}
    \def\lt{<}
    % Document parameters
    \title{Moloco\_Takehome}
    
    
    

    % Pygments definitions
    
\makeatletter
\def\PY@reset{\let\PY@it=\relax \let\PY@bf=\relax%
    \let\PY@ul=\relax \let\PY@tc=\relax%
    \let\PY@bc=\relax \let\PY@ff=\relax}
\def\PY@tok#1{\csname PY@tok@#1\endcsname}
\def\PY@toks#1+{\ifx\relax#1\empty\else%
    \PY@tok{#1}\expandafter\PY@toks\fi}
\def\PY@do#1{\PY@bc{\PY@tc{\PY@ul{%
    \PY@it{\PY@bf{\PY@ff{#1}}}}}}}
\def\PY#1#2{\PY@reset\PY@toks#1+\relax+\PY@do{#2}}

\expandafter\def\csname PY@tok@w\endcsname{\def\PY@tc##1{\textcolor[rgb]{0.73,0.73,0.73}{##1}}}
\expandafter\def\csname PY@tok@c\endcsname{\let\PY@it=\textit\def\PY@tc##1{\textcolor[rgb]{0.25,0.50,0.50}{##1}}}
\expandafter\def\csname PY@tok@cp\endcsname{\def\PY@tc##1{\textcolor[rgb]{0.74,0.48,0.00}{##1}}}
\expandafter\def\csname PY@tok@k\endcsname{\let\PY@bf=\textbf\def\PY@tc##1{\textcolor[rgb]{0.00,0.50,0.00}{##1}}}
\expandafter\def\csname PY@tok@kp\endcsname{\def\PY@tc##1{\textcolor[rgb]{0.00,0.50,0.00}{##1}}}
\expandafter\def\csname PY@tok@kt\endcsname{\def\PY@tc##1{\textcolor[rgb]{0.69,0.00,0.25}{##1}}}
\expandafter\def\csname PY@tok@o\endcsname{\def\PY@tc##1{\textcolor[rgb]{0.40,0.40,0.40}{##1}}}
\expandafter\def\csname PY@tok@ow\endcsname{\let\PY@bf=\textbf\def\PY@tc##1{\textcolor[rgb]{0.67,0.13,1.00}{##1}}}
\expandafter\def\csname PY@tok@nb\endcsname{\def\PY@tc##1{\textcolor[rgb]{0.00,0.50,0.00}{##1}}}
\expandafter\def\csname PY@tok@nf\endcsname{\def\PY@tc##1{\textcolor[rgb]{0.00,0.00,1.00}{##1}}}
\expandafter\def\csname PY@tok@nc\endcsname{\let\PY@bf=\textbf\def\PY@tc##1{\textcolor[rgb]{0.00,0.00,1.00}{##1}}}
\expandafter\def\csname PY@tok@nn\endcsname{\let\PY@bf=\textbf\def\PY@tc##1{\textcolor[rgb]{0.00,0.00,1.00}{##1}}}
\expandafter\def\csname PY@tok@ne\endcsname{\let\PY@bf=\textbf\def\PY@tc##1{\textcolor[rgb]{0.82,0.25,0.23}{##1}}}
\expandafter\def\csname PY@tok@nv\endcsname{\def\PY@tc##1{\textcolor[rgb]{0.10,0.09,0.49}{##1}}}
\expandafter\def\csname PY@tok@no\endcsname{\def\PY@tc##1{\textcolor[rgb]{0.53,0.00,0.00}{##1}}}
\expandafter\def\csname PY@tok@nl\endcsname{\def\PY@tc##1{\textcolor[rgb]{0.63,0.63,0.00}{##1}}}
\expandafter\def\csname PY@tok@ni\endcsname{\let\PY@bf=\textbf\def\PY@tc##1{\textcolor[rgb]{0.60,0.60,0.60}{##1}}}
\expandafter\def\csname PY@tok@na\endcsname{\def\PY@tc##1{\textcolor[rgb]{0.49,0.56,0.16}{##1}}}
\expandafter\def\csname PY@tok@nt\endcsname{\let\PY@bf=\textbf\def\PY@tc##1{\textcolor[rgb]{0.00,0.50,0.00}{##1}}}
\expandafter\def\csname PY@tok@nd\endcsname{\def\PY@tc##1{\textcolor[rgb]{0.67,0.13,1.00}{##1}}}
\expandafter\def\csname PY@tok@s\endcsname{\def\PY@tc##1{\textcolor[rgb]{0.73,0.13,0.13}{##1}}}
\expandafter\def\csname PY@tok@sd\endcsname{\let\PY@it=\textit\def\PY@tc##1{\textcolor[rgb]{0.73,0.13,0.13}{##1}}}
\expandafter\def\csname PY@tok@si\endcsname{\let\PY@bf=\textbf\def\PY@tc##1{\textcolor[rgb]{0.73,0.40,0.53}{##1}}}
\expandafter\def\csname PY@tok@se\endcsname{\let\PY@bf=\textbf\def\PY@tc##1{\textcolor[rgb]{0.73,0.40,0.13}{##1}}}
\expandafter\def\csname PY@tok@sr\endcsname{\def\PY@tc##1{\textcolor[rgb]{0.73,0.40,0.53}{##1}}}
\expandafter\def\csname PY@tok@ss\endcsname{\def\PY@tc##1{\textcolor[rgb]{0.10,0.09,0.49}{##1}}}
\expandafter\def\csname PY@tok@sx\endcsname{\def\PY@tc##1{\textcolor[rgb]{0.00,0.50,0.00}{##1}}}
\expandafter\def\csname PY@tok@m\endcsname{\def\PY@tc##1{\textcolor[rgb]{0.40,0.40,0.40}{##1}}}
\expandafter\def\csname PY@tok@gh\endcsname{\let\PY@bf=\textbf\def\PY@tc##1{\textcolor[rgb]{0.00,0.00,0.50}{##1}}}
\expandafter\def\csname PY@tok@gu\endcsname{\let\PY@bf=\textbf\def\PY@tc##1{\textcolor[rgb]{0.50,0.00,0.50}{##1}}}
\expandafter\def\csname PY@tok@gd\endcsname{\def\PY@tc##1{\textcolor[rgb]{0.63,0.00,0.00}{##1}}}
\expandafter\def\csname PY@tok@gi\endcsname{\def\PY@tc##1{\textcolor[rgb]{0.00,0.63,0.00}{##1}}}
\expandafter\def\csname PY@tok@gr\endcsname{\def\PY@tc##1{\textcolor[rgb]{1.00,0.00,0.00}{##1}}}
\expandafter\def\csname PY@tok@ge\endcsname{\let\PY@it=\textit}
\expandafter\def\csname PY@tok@gs\endcsname{\let\PY@bf=\textbf}
\expandafter\def\csname PY@tok@gp\endcsname{\let\PY@bf=\textbf\def\PY@tc##1{\textcolor[rgb]{0.00,0.00,0.50}{##1}}}
\expandafter\def\csname PY@tok@go\endcsname{\def\PY@tc##1{\textcolor[rgb]{0.53,0.53,0.53}{##1}}}
\expandafter\def\csname PY@tok@gt\endcsname{\def\PY@tc##1{\textcolor[rgb]{0.00,0.27,0.87}{##1}}}
\expandafter\def\csname PY@tok@err\endcsname{\def\PY@bc##1{\setlength{\fboxsep}{0pt}\fcolorbox[rgb]{1.00,0.00,0.00}{1,1,1}{\strut ##1}}}
\expandafter\def\csname PY@tok@kc\endcsname{\let\PY@bf=\textbf\def\PY@tc##1{\textcolor[rgb]{0.00,0.50,0.00}{##1}}}
\expandafter\def\csname PY@tok@kd\endcsname{\let\PY@bf=\textbf\def\PY@tc##1{\textcolor[rgb]{0.00,0.50,0.00}{##1}}}
\expandafter\def\csname PY@tok@kn\endcsname{\let\PY@bf=\textbf\def\PY@tc##1{\textcolor[rgb]{0.00,0.50,0.00}{##1}}}
\expandafter\def\csname PY@tok@kr\endcsname{\let\PY@bf=\textbf\def\PY@tc##1{\textcolor[rgb]{0.00,0.50,0.00}{##1}}}
\expandafter\def\csname PY@tok@bp\endcsname{\def\PY@tc##1{\textcolor[rgb]{0.00,0.50,0.00}{##1}}}
\expandafter\def\csname PY@tok@fm\endcsname{\def\PY@tc##1{\textcolor[rgb]{0.00,0.00,1.00}{##1}}}
\expandafter\def\csname PY@tok@vc\endcsname{\def\PY@tc##1{\textcolor[rgb]{0.10,0.09,0.49}{##1}}}
\expandafter\def\csname PY@tok@vg\endcsname{\def\PY@tc##1{\textcolor[rgb]{0.10,0.09,0.49}{##1}}}
\expandafter\def\csname PY@tok@vi\endcsname{\def\PY@tc##1{\textcolor[rgb]{0.10,0.09,0.49}{##1}}}
\expandafter\def\csname PY@tok@vm\endcsname{\def\PY@tc##1{\textcolor[rgb]{0.10,0.09,0.49}{##1}}}
\expandafter\def\csname PY@tok@sa\endcsname{\def\PY@tc##1{\textcolor[rgb]{0.73,0.13,0.13}{##1}}}
\expandafter\def\csname PY@tok@sb\endcsname{\def\PY@tc##1{\textcolor[rgb]{0.73,0.13,0.13}{##1}}}
\expandafter\def\csname PY@tok@sc\endcsname{\def\PY@tc##1{\textcolor[rgb]{0.73,0.13,0.13}{##1}}}
\expandafter\def\csname PY@tok@dl\endcsname{\def\PY@tc##1{\textcolor[rgb]{0.73,0.13,0.13}{##1}}}
\expandafter\def\csname PY@tok@s2\endcsname{\def\PY@tc##1{\textcolor[rgb]{0.73,0.13,0.13}{##1}}}
\expandafter\def\csname PY@tok@sh\endcsname{\def\PY@tc##1{\textcolor[rgb]{0.73,0.13,0.13}{##1}}}
\expandafter\def\csname PY@tok@s1\endcsname{\def\PY@tc##1{\textcolor[rgb]{0.73,0.13,0.13}{##1}}}
\expandafter\def\csname PY@tok@mb\endcsname{\def\PY@tc##1{\textcolor[rgb]{0.40,0.40,0.40}{##1}}}
\expandafter\def\csname PY@tok@mf\endcsname{\def\PY@tc##1{\textcolor[rgb]{0.40,0.40,0.40}{##1}}}
\expandafter\def\csname PY@tok@mh\endcsname{\def\PY@tc##1{\textcolor[rgb]{0.40,0.40,0.40}{##1}}}
\expandafter\def\csname PY@tok@mi\endcsname{\def\PY@tc##1{\textcolor[rgb]{0.40,0.40,0.40}{##1}}}
\expandafter\def\csname PY@tok@il\endcsname{\def\PY@tc##1{\textcolor[rgb]{0.40,0.40,0.40}{##1}}}
\expandafter\def\csname PY@tok@mo\endcsname{\def\PY@tc##1{\textcolor[rgb]{0.40,0.40,0.40}{##1}}}
\expandafter\def\csname PY@tok@ch\endcsname{\let\PY@it=\textit\def\PY@tc##1{\textcolor[rgb]{0.25,0.50,0.50}{##1}}}
\expandafter\def\csname PY@tok@cm\endcsname{\let\PY@it=\textit\def\PY@tc##1{\textcolor[rgb]{0.25,0.50,0.50}{##1}}}
\expandafter\def\csname PY@tok@cpf\endcsname{\let\PY@it=\textit\def\PY@tc##1{\textcolor[rgb]{0.25,0.50,0.50}{##1}}}
\expandafter\def\csname PY@tok@c1\endcsname{\let\PY@it=\textit\def\PY@tc##1{\textcolor[rgb]{0.25,0.50,0.50}{##1}}}
\expandafter\def\csname PY@tok@cs\endcsname{\let\PY@it=\textit\def\PY@tc##1{\textcolor[rgb]{0.25,0.50,0.50}{##1}}}

\def\PYZbs{\char`\\}
\def\PYZus{\char`\_}
\def\PYZob{\char`\{}
\def\PYZcb{\char`\}}
\def\PYZca{\char`\^}
\def\PYZam{\char`\&}
\def\PYZlt{\char`\<}
\def\PYZgt{\char`\>}
\def\PYZsh{\char`\#}
\def\PYZpc{\char`\%}
\def\PYZdl{\char`\$}
\def\PYZhy{\char`\-}
\def\PYZsq{\char`\'}
\def\PYZdq{\char`\"}
\def\PYZti{\char`\~}
% for compatibility with earlier versions
\def\PYZat{@}
\def\PYZlb{[}
\def\PYZrb{]}
\makeatother


    % Exact colors from NB
    \definecolor{incolor}{rgb}{0.0, 0.0, 0.5}
    \definecolor{outcolor}{rgb}{0.545, 0.0, 0.0}



    
    % Prevent overflowing lines due to hard-to-break entities
    \sloppy 
    % Setup hyperref package
    \hypersetup{
      breaklinks=true,  % so long urls are correctly broken across lines
      colorlinks=true,
      urlcolor=urlcolor,
      linkcolor=linkcolor,
      citecolor=citecolor,
      }
    % Slightly bigger margins than the latex defaults
    
    \geometry{verbose,tmargin=1in,bmargin=1in,lmargin=1in,rmargin=1in}
    
    

    \begin{document}
    
    
    \maketitle
    
    

    
    Moloco Takehome Exam

1.Read Data

    \begin{Verbatim}[commandchars=\\\{\}]
{\color{incolor}In [{\color{incolor}1}]:} \PY{k+kn}{import} \PY{n+nn}{numpy} \PY{k}{as} \PY{n+nn}{np}
        \PY{k+kn}{import} \PY{n+nn}{pandas} \PY{k}{as} \PY{n+nn}{pd}
        \PY{k+kn}{import} \PY{n+nn}{warnings}
        \PY{n}{warnings}\PY{o}{.}\PY{n}{filterwarnings}\PY{p}{(}\PY{l+s+s2}{\PYZdq{}}\PY{l+s+s2}{ignore}\PY{l+s+s2}{\PYZdq{}}\PY{p}{)}
        \PY{n}{User\PYZus{}data} \PY{o}{=} \PY{n}{pd}\PY{o}{.}\PY{n}{read\PYZus{}excel}\PY{p}{(}\PY{l+s+s2}{\PYZdq{}}\PY{l+s+s2}{Adops \PYZam{} Data Scientist Sample Data.xlsx}\PY{l+s+s2}{\PYZdq{}}\PY{p}{,} \PY{n}{sheet\PYZus{}name}\PY{o}{=}\PY{l+s+s2}{\PYZdq{}}\PY{l+s+s2}{Q1 Analytics}\PY{l+s+s2}{\PYZdq{}}\PY{p}{,} \PY{n}{header}\PY{o}{=}\PY{l+m+mi}{0}\PY{p}{)}
\end{Verbatim}


    2.Transform timestamps

    \begin{Verbatim}[commandchars=\\\{\}]
{\color{incolor}In [{\color{incolor}2}]:} \PY{k+kn}{import} \PY{n+nn}{time}
        \PY{k+kn}{import} \PY{n+nn}{datetime}
        \PY{n}{reformed} \PY{o}{=} \PY{p}{[}\PY{p}{]}
        \PY{k}{for} \PY{n}{date} \PY{o+ow}{in} \PY{n}{User\PYZus{}data}\PY{p}{[}\PY{l+s+s2}{\PYZdq{}}\PY{l+s+s2}{ts}\PY{l+s+s2}{\PYZdq{}}\PY{p}{]}\PY{p}{:}
            \PY{k}{if} \PY{n+nb}{isinstance}\PY{p}{(}\PY{n}{date}\PY{p}{,}\PY{n+nb}{str}\PY{p}{)}\PY{p}{:}
                \PY{n}{date} \PY{o}{=} \PY{n}{time}\PY{o}{.}\PY{n}{strptime}\PY{p}{(}\PY{n}{date}\PY{p}{,} \PY{l+s+s2}{\PYZdq{}}\PY{l+s+s2}{\PYZpc{}}\PY{l+s+s2}{Y\PYZhy{}}\PY{l+s+s2}{\PYZpc{}}\PY{l+s+s2}{m\PYZhy{}}\PY{l+s+si}{\PYZpc{}d}\PY{l+s+s2}{ }\PY{l+s+s2}{\PYZpc{}}\PY{l+s+s2}{H:}\PY{l+s+s2}{\PYZpc{}}\PY{l+s+s2}{M:}\PY{l+s+s2}{\PYZpc{}}\PY{l+s+s2}{S}\PY{l+s+s2}{\PYZdq{}}\PY{p}{)}
                \PY{n}{reformed}\PY{o}{.}\PY{n}{append}\PY{p}{(}\PY{n}{date}\PY{p}{)}
        \PY{n}{User\PYZus{}data}\PY{p}{[}\PY{l+s+s2}{\PYZdq{}}\PY{l+s+s2}{ts}\PY{l+s+s2}{\PYZdq{}}\PY{p}{]}\PY{o}{=}\PY{n}{pd}\PY{o}{.}\PY{n}{Series}\PY{p}{(}\PY{n}{reformed}\PY{p}{)}
\end{Verbatim}


    \begin{enumerate}
\def\labelenumi{\arabic{enumi}.}
\setcounter{enumi}{2}
\tightlist
\item
  Answer to Part 1 questions

  \begin{enumerate}
  \def\labelenumii{\roman{enumii}.}
  \tightlist
  \item
    Site with most unique users

    Consider only the rows with country\_id = "BDV" (there are 844 such
    rows). For each site\_id, we can compute the number of unique
    user\_id's found in these 844 rows. Which site\_id has the largest
    number of unique users?
  \end{enumerate}
\end{enumerate}

    \begin{Verbatim}[commandchars=\\\{\}]
{\color{incolor}In [{\color{incolor}3}]:} \PY{n}{bdv\PYZus{}data} \PY{o}{=} \PY{n}{User\PYZus{}data}\PY{p}{[}\PY{n}{User\PYZus{}data}\PY{p}{[}\PY{l+s+s2}{\PYZdq{}}\PY{l+s+s2}{country\PYZus{}id}\PY{l+s+s2}{\PYZdq{}}\PY{p}{]}\PY{o}{==}\PY{l+s+s2}{\PYZdq{}}\PY{l+s+s2}{BDV}\PY{l+s+s2}{\PYZdq{}}\PY{p}{]}
        \PY{n}{bdv\PYZus{}data}\PY{o}{.}\PY{n}{groupby}\PY{p}{(}\PY{l+s+s2}{\PYZdq{}}\PY{l+s+s2}{site\PYZus{}id}\PY{l+s+s2}{\PYZdq{}}\PY{p}{)}\PY{o}{.}\PY{n}{count}\PY{p}{(}\PY{p}{)}\PY{o}{.}\PY{n}{sort\PYZus{}values}\PY{p}{(}\PY{l+s+s2}{\PYZdq{}}\PY{l+s+s2}{user\PYZus{}id}\PY{l+s+s2}{\PYZdq{}}\PY{p}{,} \PY{n}{ascending}\PY{o}{=}\PY{k+kc}{False}\PY{p}{)}
\end{Verbatim}


\begin{Verbatim}[commandchars=\\\{\}]
{\color{outcolor}Out[{\color{outcolor}3}]:}           ts  user\_id  country\_id
        site\_id                          
        5NPAU    717      717         717
        N0OTG    122      122         122
        3POLC      5        5           5
\end{Verbatim}
            
    The answer is "5NPAU" with 717 unique users.

    \begin{enumerate}
\def\labelenumi{\roman{enumi}.}
\setcounter{enumi}{1}
\tightlist
\item
  Four users \& which sites they (each) visited more than 10 times

  Between 2019-02-03 00:00:00 and 2019-02-04 23:59:59, there are four
  users who visited a certain site more than 10 times. Find these four
  users \& which sites they (each) visited more than 10 times. (Simply
  provides four triples in the form (user\_id, site\_id, number of
  visits) in the box below.)
\end{enumerate}

    \begin{Verbatim}[commandchars=\\\{\}]
{\color{incolor}In [{\color{incolor}4}]:} \PY{n}{start} \PY{o}{=} \PY{n}{time}\PY{o}{.}\PY{n}{strptime}\PY{p}{(}\PY{l+s+s2}{\PYZdq{}}\PY{l+s+s2}{2019\PYZhy{}02\PYZhy{}03 00:00:00}\PY{l+s+s2}{\PYZdq{}}\PY{p}{,} \PY{l+s+s2}{\PYZdq{}}\PY{l+s+s2}{\PYZpc{}}\PY{l+s+s2}{Y\PYZhy{}}\PY{l+s+s2}{\PYZpc{}}\PY{l+s+s2}{m\PYZhy{}}\PY{l+s+si}{\PYZpc{}d}\PY{l+s+s2}{ }\PY{l+s+s2}{\PYZpc{}}\PY{l+s+s2}{H:}\PY{l+s+s2}{\PYZpc{}}\PY{l+s+s2}{M:}\PY{l+s+s2}{\PYZpc{}}\PY{l+s+s2}{S}\PY{l+s+s2}{\PYZdq{}}\PY{p}{)}
        \PY{n}{end} \PY{o}{=} \PY{n}{time}\PY{o}{.}\PY{n}{strptime}\PY{p}{(}\PY{l+s+s2}{\PYZdq{}}\PY{l+s+s2}{2019\PYZhy{}02\PYZhy{}04 23:59:59}\PY{l+s+s2}{\PYZdq{}}\PY{p}{,} \PY{l+s+s2}{\PYZdq{}}\PY{l+s+s2}{\PYZpc{}}\PY{l+s+s2}{Y\PYZhy{}}\PY{l+s+s2}{\PYZpc{}}\PY{l+s+s2}{m\PYZhy{}}\PY{l+s+si}{\PYZpc{}d}\PY{l+s+s2}{ }\PY{l+s+s2}{\PYZpc{}}\PY{l+s+s2}{H:}\PY{l+s+s2}{\PYZpc{}}\PY{l+s+s2}{M:}\PY{l+s+s2}{\PYZpc{}}\PY{l+s+s2}{S}\PY{l+s+s2}{\PYZdq{}}\PY{p}{)}
        \PY{n}{time\PYZus{}window} \PY{o}{=} \PY{n}{User\PYZus{}data}\PY{p}{[}\PY{p}{(}\PY{n}{User\PYZus{}data}\PY{p}{[}\PY{l+s+s2}{\PYZdq{}}\PY{l+s+s2}{ts}\PY{l+s+s2}{\PYZdq{}}\PY{p}{]}\PY{o}{\PYZgt{}}\PY{o}{=}\PY{n}{start}\PY{p}{)} \PY{o}{\PYZam{}} \PY{p}{(}\PY{n}{User\PYZus{}data}\PY{p}{[}\PY{l+s+s2}{\PYZdq{}}\PY{l+s+s2}{ts}\PY{l+s+s2}{\PYZdq{}}\PY{p}{]}\PY{o}{\PYZlt{}}\PY{o}{=}\PY{n}{end}\PY{p}{)}\PY{p}{]}
        \PY{n}{grouped} \PY{o}{=} \PY{n}{time\PYZus{}window}\PY{o}{.}\PY{n}{groupby}\PY{p}{(}\PY{p}{[}\PY{l+s+s2}{\PYZdq{}}\PY{l+s+s2}{user\PYZus{}id}\PY{l+s+s2}{\PYZdq{}}\PY{p}{,}\PY{l+s+s2}{\PYZdq{}}\PY{l+s+s2}{site\PYZus{}id}\PY{l+s+s2}{\PYZdq{}}\PY{p}{]}\PY{p}{)}\PY{o}{.}\PY{n}{count}\PY{p}{(}\PY{p}{)}
        \PY{n}{grouped}\PY{p}{[}\PY{n}{grouped}\PY{p}{[}\PY{l+s+s2}{\PYZdq{}}\PY{l+s+s2}{ts}\PY{l+s+s2}{\PYZdq{}}\PY{p}{]}\PY{o}{\PYZgt{}}\PY{o}{=}\PY{l+m+mi}{10}\PY{p}{]}
\end{Verbatim}


\begin{Verbatim}[commandchars=\\\{\}]
{\color{outcolor}Out[{\color{outcolor}4}]:}                  ts  country\_id
        user\_id site\_id                
        LC06C3  N0OTG    25          25
        LC3A59  N0OTG    26          26
        LC3C7E  3POLC    15          15
        LC3C9D  N0OTG    17          17
\end{Verbatim}
            
    The answer is ("LC06C3", "N0OTG",25) , ("LC3A59", "N0OTG",26),
("LC3C7E", "3POLC",15), ("LC3C9D", "N0OTG",17)

    \begin{enumerate}
\def\labelenumi{\roman{enumi}.}
\setcounter{enumi}{2}
\tightlist
\item
  Top three last visit sites

  For each site, compute the unique number of users whose last visit
  (found in the original data set) was to that site. For instance, user
  "LC3561"'s last visit is to "N0OTG" based on timestamp data. Based on
  this measure, what are top three sites? (hint: site "3POLC" is ranked
  at 5th with 28 users whose last visit in the data set was to 3POLC;
  simply provide three pairs in the form (site\_id, number of users).)
\end{enumerate}

    \begin{Verbatim}[commandchars=\\\{\}]
{\color{incolor}In [{\color{incolor}5}]:} \PY{n}{user\PYZus{}grouped} \PY{o}{=} \PY{n}{User\PYZus{}data}\PY{o}{.}\PY{n}{groupby}\PY{p}{(}\PY{l+s+s2}{\PYZdq{}}\PY{l+s+s2}{user\PYZus{}id}\PY{l+s+s2}{\PYZdq{}}\PY{p}{)}
\end{Verbatim}


    \begin{Verbatim}[commandchars=\\\{\}]
{\color{incolor}In [{\color{incolor}6}]:} \PY{k}{def} \PY{n+nf}{getFirstVisit}\PY{p}{(}\PY{n}{user}\PY{p}{,}\PY{n}{group}\PY{p}{)}\PY{p}{:}
            \PY{n}{group} \PY{o}{=} \PY{n}{group}\PY{o}{.}\PY{n}{sort\PYZus{}values}\PY{p}{(}\PY{l+s+s2}{\PYZdq{}}\PY{l+s+s2}{ts}\PY{l+s+s2}{\PYZdq{}}\PY{p}{,}\PY{n}{ascending}\PY{o}{=}\PY{k+kc}{True}\PY{p}{)}
            \PY{k}{return} \PY{n}{group}\PY{o}{.}\PY{n}{iloc}\PY{p}{[}\PY{l+m+mi}{0}\PY{p}{,}\PY{l+m+mi}{3}\PY{p}{]}
        \PY{k}{def} \PY{n+nf}{getLastVisit}\PY{p}{(}\PY{n}{user}\PY{p}{,}\PY{n}{group}\PY{p}{)}\PY{p}{:}
            \PY{n}{group} \PY{o}{=} \PY{n}{group}\PY{o}{.}\PY{n}{sort\PYZus{}values}\PY{p}{(}\PY{l+s+s2}{\PYZdq{}}\PY{l+s+s2}{ts}\PY{l+s+s2}{\PYZdq{}}\PY{p}{,}\PY{n}{ascending}\PY{o}{=}\PY{k+kc}{False}\PY{p}{)}
            \PY{k}{return} \PY{n}{group}\PY{o}{.}\PY{n}{iloc}\PY{p}{[}\PY{l+m+mi}{0}\PY{p}{,}\PY{l+m+mi}{3}\PY{p}{]}
\end{Verbatim}


    \begin{Verbatim}[commandchars=\\\{\}]
{\color{incolor}In [{\color{incolor}7}]:} \PY{n}{first\PYZus{}sites} \PY{o}{=} \PY{p}{[}\PY{p}{]}
        \PY{n}{last\PYZus{}sites} \PY{o}{=} \PY{p}{[}\PY{p}{]}
        \PY{n}{users} \PY{o}{=} \PY{p}{[}\PY{p}{]}
        \PY{k}{for} \PY{n}{user}\PY{p}{,} \PY{n}{group} \PY{o+ow}{in} \PY{n}{user\PYZus{}grouped}\PY{p}{:}
            \PY{n}{users}\PY{o}{.}\PY{n}{append}\PY{p}{(}\PY{n}{user}\PY{p}{)}
            \PY{n}{first\PYZus{}sites}\PY{o}{.}\PY{n}{append}\PY{p}{(}\PY{n}{getFirstVisit}\PY{p}{(}\PY{n}{user}\PY{p}{,}\PY{n}{group}\PY{p}{)}\PY{p}{)}
            \PY{n}{last\PYZus{}sites}\PY{o}{.}\PY{n}{append}\PY{p}{(}\PY{n}{getLastVisit}\PY{p}{(}\PY{n}{user}\PY{p}{,}\PY{n}{group}\PY{p}{)}\PY{p}{)}
        \PY{n}{result} \PY{o}{=} \PY{n}{pd}\PY{o}{.}\PY{n}{DataFrame}\PY{p}{(}\PY{p}{[}\PY{n}{pd}\PY{o}{.}\PY{n}{Series}\PY{p}{(}\PY{n}{users}\PY{p}{)}\PY{p}{,}\PY{n}{pd}\PY{o}{.}\PY{n}{Series}\PY{p}{(}\PY{n}{first\PYZus{}sites}\PY{p}{)}\PY{p}{,}\PY{n}{pd}\PY{o}{.}\PY{n}{Series}\PY{p}{(}\PY{n}{last\PYZus{}sites}\PY{p}{)}\PY{p}{]}\PY{p}{,}\PY{n}{index}\PY{o}{=}\PY{p}{[}\PY{l+s+s2}{\PYZdq{}}\PY{l+s+s2}{users}\PY{l+s+s2}{\PYZdq{}}\PY{p}{,}\PY{l+s+s2}{\PYZdq{}}\PY{l+s+s2}{first visit site}\PY{l+s+s2}{\PYZdq{}}\PY{p}{,} \PY{l+s+s2}{\PYZdq{}}\PY{l+s+s2}{last visit site}\PY{l+s+s2}{\PYZdq{}}\PY{p}{]}\PY{p}{)}\PY{o}{.}\PY{n}{T}
\end{Verbatim}


    \begin{Verbatim}[commandchars=\\\{\}]
{\color{incolor}In [{\color{incolor}8}]:} \PY{n}{result}\PY{o}{.}\PY{n}{groupby}\PY{p}{(}\PY{l+s+s2}{\PYZdq{}}\PY{l+s+s2}{last visit site}\PY{l+s+s2}{\PYZdq{}}\PY{p}{)}\PY{o}{.}\PY{n}{count}\PY{p}{(}\PY{p}{)}\PY{o}{.}\PY{n}{sort\PYZus{}values}\PY{p}{(}\PY{l+s+s2}{\PYZdq{}}\PY{l+s+s2}{users}\PY{l+s+s2}{\PYZdq{}}\PY{p}{,}\PY{n}{ascending}\PY{o}{=}\PY{k+kc}{False}\PY{p}{)}
\end{Verbatim}


\begin{Verbatim}[commandchars=\\\{\}]
{\color{outcolor}Out[{\color{outcolor}8}]:}                  users  first visit site
        last visit site                         
        5NPAU              992               992
        N0OTG              561               561
        QGO3G              289               289
        GVOFK               42                42
        3POLC               28                28
        RT9Z6                2                 2
        EUZ/Q                1                 1
        JSUUP                1                 1
\end{Verbatim}
            
    The answer is ("5NPAU", 992), ("N0OTG", 561), ("QGO3G", 289)

    \begin{enumerate}
\def\labelenumi{\roman{enumi}.}
\setcounter{enumi}{3}
\tightlist
\item
  The number of users whose first/last visits are to the same website

  For each user, determine the first site he/she visited and the last
  site he/she visited based on the timestamp data. Compute the number of
  users whose first/last visits are to the same website. What is the
  number?
\end{enumerate}

    \begin{Verbatim}[commandchars=\\\{\}]
{\color{incolor}In [{\color{incolor}9}]:} \PY{n}{result}\PY{p}{[}\PY{n}{result}\PY{p}{[}\PY{l+s+s2}{\PYZdq{}}\PY{l+s+s2}{first visit site}\PY{l+s+s2}{\PYZdq{}}\PY{p}{]}\PY{o}{==}\PY{n}{result}\PY{p}{[}\PY{l+s+s2}{\PYZdq{}}\PY{l+s+s2}{last visit site}\PY{l+s+s2}{\PYZdq{}}\PY{p}{]}\PY{p}{]}\PY{o}{.}\PY{n}{shape}\PY{p}{[}\PY{l+m+mi}{0}\PY{p}{]}
\end{Verbatim}


\begin{Verbatim}[commandchars=\\\{\}]
{\color{outcolor}Out[{\color{outcolor}9}]:} 1670
\end{Verbatim}
            
    \begin{enumerate}
\def\labelenumi{\arabic{enumi}.}
\setcounter{enumi}{3}
\tightlist
\item
  Part 2 Regression model
\end{enumerate}

    The data contains 300 rows and 3 columns (from the left, A, B, and C).
Please build a good regression model which explains column C by a
function of A and B.

Since column A, B and C are all numerical values, I used linear
regresion with its loss fuction as MSE here.

    \begin{Verbatim}[commandchars=\\\{\}]
{\color{incolor}In [{\color{incolor}10}]:} \PY{k+kn}{import} \PY{n+nn}{numpy} \PY{k}{as} \PY{n+nn}{np}
         \PY{k+kn}{import} \PY{n+nn}{pandas} \PY{k}{as} \PY{n+nn}{pd}
         \PY{k+kn}{import} \PY{n+nn}{scipy}
         \PY{k+kn}{from} \PY{n+nn}{scipy} \PY{k}{import} \PY{n}{optimize}
         \PY{k+kn}{import} \PY{n+nn}{warnings}
         \PY{n}{warnings}\PY{o}{.}\PY{n}{filterwarnings}\PY{p}{(}\PY{l+s+s2}{\PYZdq{}}\PY{l+s+s2}{ignore}\PY{l+s+s2}{\PYZdq{}}\PY{p}{)}
         \PY{n}{data} \PY{o}{=} \PY{n}{pd}\PY{o}{.}\PY{n}{read\PYZus{}excel}\PY{p}{(}\PY{l+s+s2}{\PYZdq{}}\PY{l+s+s2}{Adops \PYZam{} Data Scientist Sample Data.xlsx}\PY{l+s+s2}{\PYZdq{}}\PY{p}{,} \PY{n}{sheet\PYZus{}name}\PY{o}{=}\PY{l+s+s2}{\PYZdq{}}\PY{l+s+s2}{Q2 Regression}\PY{l+s+s2}{\PYZdq{}}\PY{p}{,}\PY{n}{header}\PY{o}{=}\PY{k+kc}{None}\PY{p}{)}
         \PY{n}{data}\PY{o}{.}\PY{n}{columns} \PY{o}{=} \PY{p}{[}\PY{l+s+s2}{\PYZdq{}}\PY{l+s+s2}{A}\PY{l+s+s2}{\PYZdq{}}\PY{p}{,}\PY{l+s+s2}{\PYZdq{}}\PY{l+s+s2}{B}\PY{l+s+s2}{\PYZdq{}}\PY{p}{,}\PY{l+s+s2}{\PYZdq{}}\PY{l+s+s2}{C}\PY{l+s+s2}{\PYZdq{}}\PY{p}{]}
         \PY{n}{data}\PY{o}{.}\PY{n}{shape}
\end{Verbatim}


\begin{Verbatim}[commandchars=\\\{\}]
{\color{outcolor}Out[{\color{outcolor}10}]:} (300, 3)
\end{Verbatim}
            
    \begin{enumerate}
\def\labelenumi{\roman{enumi}.}
\tightlist
\item
  Detect Null values and outliers.

  There's no Null values in our dataset.

  My criterion for detecting outlier (high leverage point) is: outside
  the interval of mean ± (2.33*standard deviation).

  If we assume all the 3 variables has enough samples(Central Limit
  Theorem), 1\% of the samples of each variable would be considered as
  outliers. The threshold could also be adjusted.
\end{enumerate}

    \begin{Verbatim}[commandchars=\\\{\}]
{\color{incolor}In [{\color{incolor}11}]:} \PY{n}{data} \PY{o}{=} \PY{n}{data}\PY{o}{.}\PY{n}{dropna}\PY{p}{(}\PY{p}{)}
\end{Verbatim}


    \begin{Verbatim}[commandchars=\\\{\}]
{\color{incolor}In [{\color{incolor}12}]:} \PY{k}{def} \PY{n+nf}{outlierDetector}\PY{p}{(}\PY{n}{dataseries}\PY{p}{)}\PY{p}{:}
             \PY{n}{mean} \PY{o}{=} \PY{n}{dataseries}\PY{o}{.}\PY{n}{mean}\PY{p}{(}\PY{p}{)}
             \PY{n}{std} \PY{o}{=} \PY{n}{dataseries}\PY{o}{.}\PY{n}{std}\PY{p}{(}\PY{p}{)}
             \PY{n}{lower\PYZus{}bond} \PY{o}{=} \PY{n}{mean} \PY{o}{\PYZhy{}} \PY{p}{(}\PY{l+m+mf}{2.33}\PY{o}{*}\PY{n}{std}\PY{p}{)}
             \PY{n}{upper\PYZus{}bond} \PY{o}{=} \PY{n}{mean} \PY{o}{+} \PY{p}{(}\PY{l+m+mf}{2.33}\PY{o}{*}\PY{n}{std}\PY{p}{)}
             \PY{n}{outliers} \PY{o}{=} \PY{p}{[}\PY{p}{]}
             \PY{k}{for} \PY{n}{record} \PY{o+ow}{in} \PY{n}{dataseries}\PY{p}{:}
                 \PY{k}{if} \PY{n}{record} \PY{o}{\PYZlt{}} \PY{n}{lower\PYZus{}bond} \PY{o+ow}{or} \PY{n}{record} \PY{o}{\PYZgt{}} \PY{n}{upper\PYZus{}bond}\PY{p}{:}
                     \PY{n}{outliers}\PY{o}{.}\PY{n}{append}\PY{p}{(}\PY{n}{record}\PY{p}{)}
             \PY{k}{return} \PY{n}{outliers}
\end{Verbatim}


    \begin{Verbatim}[commandchars=\\\{\}]
{\color{incolor}In [{\color{incolor}13}]:} \PY{n}{data} \PY{o}{=} \PY{n}{data}\PY{p}{[}\PY{o}{\PYZti{}}\PY{n}{data}\PY{p}{[}\PY{l+s+s2}{\PYZdq{}}\PY{l+s+s2}{A}\PY{l+s+s2}{\PYZdq{}}\PY{p}{]}\PY{o}{.}\PY{n}{isin}\PY{p}{(}\PY{n}{outlierDetector}\PY{p}{(}\PY{n}{data}\PY{p}{[}\PY{l+s+s2}{\PYZdq{}}\PY{l+s+s2}{A}\PY{l+s+s2}{\PYZdq{}}\PY{p}{]}\PY{p}{)}\PY{p}{)} \PY{o}{\PYZam{}} \PY{o}{\PYZti{}}\PY{n}{data}\PY{p}{[}\PY{l+s+s2}{\PYZdq{}}\PY{l+s+s2}{B}\PY{l+s+s2}{\PYZdq{}}\PY{p}{]}\PY{o}{.}\PY{n}{isin}\PY{p}{(}\PY{n}{outlierDetector}\PY{p}{(}\PY{n}{data}\PY{p}{[}\PY{l+s+s2}{\PYZdq{}}\PY{l+s+s2}{B}\PY{l+s+s2}{\PYZdq{}}\PY{p}{]}\PY{p}{)}\PY{p}{)} \PY{o}{\PYZam{}} \PY{o}{\PYZti{}}\PY{n}{data}\PY{p}{[}\PY{l+s+s2}{\PYZdq{}}\PY{l+s+s2}{C}\PY{l+s+s2}{\PYZdq{}}\PY{p}{]}\PY{o}{.}\PY{n}{isin}\PY{p}{(}\PY{n}{outlierDetector}\PY{p}{(}\PY{n}{data}\PY{p}{[}\PY{l+s+s2}{\PYZdq{}}\PY{l+s+s2}{C}\PY{l+s+s2}{\PYZdq{}}\PY{p}{]}\PY{p}{)}\PY{p}{)}\PY{p}{]}
         \PY{n}{data}\PY{o}{.}\PY{n}{shape}
\end{Verbatim}


\begin{Verbatim}[commandchars=\\\{\}]
{\color{outcolor}Out[{\color{outcolor}13}]:} (297, 3)
\end{Verbatim}
            
    \begin{enumerate}
\def\labelenumi{\roman{enumi}.}
\setcounter{enumi}{1}
\tightlist
\item
  Do the linear regression

  We have our dependant variable with 2 independant variables, the
  result of our regression should looks like a plane in the space.
\end{enumerate}

    \begin{Verbatim}[commandchars=\\\{\}]
{\color{incolor}In [{\color{incolor}14}]:} \PY{k}{def} \PY{n+nf}{MSE}\PY{p}{(}\PY{n}{b}\PY{p}{)}\PY{p}{:}
             \PY{n}{x0} \PY{o}{=} \PY{n}{np}\PY{o}{.}\PY{n}{array}\PY{p}{(}\PY{n}{data}\PY{p}{[}\PY{l+s+s2}{\PYZdq{}}\PY{l+s+s2}{A}\PY{l+s+s2}{\PYZdq{}}\PY{p}{]}\PY{p}{)}
             \PY{n}{x1} \PY{o}{=} \PY{n}{np}\PY{o}{.}\PY{n}{array}\PY{p}{(}\PY{n}{data}\PY{p}{[}\PY{l+s+s2}{\PYZdq{}}\PY{l+s+s2}{B}\PY{l+s+s2}{\PYZdq{}}\PY{p}{]}\PY{p}{)}
             \PY{n}{y} \PY{o}{=}  \PY{n}{np}\PY{o}{.}\PY{n}{array}\PY{p}{(}\PY{n}{data}\PY{p}{[}\PY{l+s+s2}{\PYZdq{}}\PY{l+s+s2}{C}\PY{l+s+s2}{\PYZdq{}}\PY{p}{]}\PY{p}{)}
             \PY{n}{pred} \PY{o}{=}\PY{n}{b}\PY{p}{[}\PY{l+m+mi}{0}\PY{p}{]} \PY{o}{+} \PY{n}{b}\PY{p}{[}\PY{l+m+mi}{1}\PY{p}{]}\PY{o}{*}\PY{n}{x0} \PY{o}{+} \PY{n}{b}\PY{p}{[}\PY{l+m+mi}{2}\PY{p}{]}\PY{o}{*}\PY{n}{x1}
             \PY{n}{mse} \PY{o}{=} \PY{n+nb}{sum}\PY{p}{(}\PY{n+nb}{list}\PY{p}{(}\PY{n+nb}{map}\PY{p}{(}\PY{k}{lambda} \PY{n}{x}\PY{p}{:} \PY{n}{x}\PY{o}{*}\PY{o}{*}\PY{l+m+mi}{2}\PY{p}{,} \PY{n}{y} \PY{o}{\PYZhy{}} \PY{n}{pred}\PY{p}{)}\PY{p}{)}\PY{p}{)}
             \PY{k}{return} \PY{n}{mse}
\end{Verbatim}


    \begin{Verbatim}[commandchars=\\\{\}]
{\color{incolor}In [{\color{incolor}15}]:} \PY{n}{result} \PY{o}{=} \PY{n}{scipy}\PY{o}{.}\PY{n}{optimize}\PY{o}{.}\PY{n}{minimize}\PY{p}{(}\PY{n}{MSE}\PY{p}{,} \PY{p}{[}\PY{l+m+mi}{1}\PY{p}{,}\PY{l+m+mi}{1}\PY{p}{,}\PY{l+m+mi}{1}\PY{p}{]}\PY{p}{)}
         \PY{n}{result}\PY{o}{.}\PY{n}{x}
\end{Verbatim}


\begin{Verbatim}[commandchars=\\\{\}]
{\color{outcolor}Out[{\color{outcolor}15}]:} array([ 25.03925411,  -1.09164429, -15.0198942 ])
\end{Verbatim}
            
    \begin{Verbatim}[commandchars=\\\{\}]
{\color{incolor}In [{\color{incolor}16}]:} \PY{n+nb}{print}\PY{p}{(}\PY{l+s+s2}{\PYZdq{}}\PY{l+s+s2}{The relationship between A, B and C is: C = }\PY{l+s+si}{\PYZpc{} 5.2f}\PY{l+s+s2}{ + (}\PY{l+s+si}{\PYZpc{} 5.2f}\PY{l+s+s2}{ * A) + (}\PY{l+s+si}{\PYZpc{} 5.2f}\PY{l+s+s2}{ * B)}\PY{l+s+s2}{\PYZdq{}}\PY{o}{\PYZpc{}}\PY{k}{tuple}(result.x))
\end{Verbatim}


    \begin{Verbatim}[commandchars=\\\{\}]
The relationship between A, B and C is: C =  25.04 + (-1.09 * A) + (-15.02 * B)

    \end{Verbatim}

    \begin{enumerate}
\def\labelenumi{\roman{enumi}.}
\setcounter{enumi}{2}
\tightlist
\item
  Visualization
\end{enumerate}

    \begin{Verbatim}[commandchars=\\\{\}]
{\color{incolor}In [{\color{incolor}17}]:} \PY{n}{X}\PY{p}{,} \PY{n}{Y} \PY{o}{=} \PY{n}{np}\PY{o}{.}\PY{n}{meshgrid}\PY{p}{(}\PY{n}{data}\PY{p}{[}\PY{l+s+s2}{\PYZdq{}}\PY{l+s+s2}{A}\PY{l+s+s2}{\PYZdq{}}\PY{p}{]}\PY{p}{,} \PY{n}{data}\PY{p}{[}\PY{l+s+s2}{\PYZdq{}}\PY{l+s+s2}{B}\PY{l+s+s2}{\PYZdq{}}\PY{p}{]}\PY{p}{)}
         \PY{n}{Z} \PY{o}{=} \PY{n}{result}\PY{o}{.}\PY{n}{x}\PY{p}{[}\PY{l+m+mi}{0}\PY{p}{]} \PY{o}{+} \PY{n}{result}\PY{o}{.}\PY{n}{x}\PY{p}{[}\PY{l+m+mi}{1}\PY{p}{]}\PY{o}{*}\PY{n}{X} \PY{o}{+}\PY{n}{result}\PY{o}{.}\PY{n}{x}\PY{p}{[}\PY{l+m+mi}{2}\PY{p}{]}\PY{o}{*}\PY{n}{Y}
\end{Verbatim}


    \begin{Verbatim}[commandchars=\\\{\}]
{\color{incolor}In [{\color{incolor}18}]:} \PY{k+kn}{import} \PY{n+nn}{matplotlib}\PY{n+nn}{.}\PY{n+nn}{pyplot} \PY{k}{as} \PY{n+nn}{plt}
         \PY{k+kn}{from} \PY{n+nn}{mpl\PYZus{}toolkits}\PY{n+nn}{.}\PY{n+nn}{mplot3d} \PY{k}{import} \PY{n}{Axes3D}
         \PY{o}{\PYZpc{}}\PY{k}{matplotlib} inline
         \PY{n}{fig} \PY{o}{=} \PY{n}{plt}\PY{o}{.}\PY{n}{figure}\PY{p}{(}\PY{n}{figsize}\PY{o}{=}\PY{p}{[}\PY{l+m+mi}{10}\PY{p}{,}\PY{l+m+mi}{10}\PY{p}{]}\PY{p}{)}
         \PY{n}{ax} \PY{o}{=} \PY{n}{Axes3D}\PY{p}{(}\PY{n}{fig}\PY{p}{)}
         \PY{n}{ax}\PY{o}{.}\PY{n}{scatter}\PY{p}{(}\PY{n}{data}\PY{p}{[}\PY{l+s+s2}{\PYZdq{}}\PY{l+s+s2}{A}\PY{l+s+s2}{\PYZdq{}}\PY{p}{]}\PY{p}{,}\PY{n}{data}\PY{p}{[}\PY{l+s+s2}{\PYZdq{}}\PY{l+s+s2}{B}\PY{l+s+s2}{\PYZdq{}}\PY{p}{]}\PY{p}{,}\PY{n}{data}\PY{p}{[}\PY{l+s+s2}{\PYZdq{}}\PY{l+s+s2}{C}\PY{l+s+s2}{\PYZdq{}}\PY{p}{]}\PY{p}{,}\PY{n}{c}\PY{o}{=}\PY{l+s+s2}{\PYZdq{}}\PY{l+s+s2}{y}\PY{l+s+s2}{\PYZdq{}}\PY{p}{)}
         \PY{n}{ax}\PY{o}{.}\PY{n}{plot\PYZus{}surface}\PY{p}{(}\PY{n}{X}\PY{p}{,}\PY{n}{Y}\PY{p}{,}\PY{n}{Z}\PY{p}{,} \PY{n}{cmap}\PY{o}{=}\PY{l+s+s2}{\PYZdq{}}\PY{l+s+s2}{Blues}\PY{l+s+s2}{\PYZdq{}}\PY{p}{)}
\end{Verbatim}


\begin{Verbatim}[commandchars=\\\{\}]
{\color{outcolor}Out[{\color{outcolor}18}]:} <mpl\_toolkits.mplot3d.art3d.Poly3DCollection at 0x1175d3278>
\end{Verbatim}
            
    \begin{center}
    \adjustimage{max size={0.9\linewidth}{0.9\paperheight}}{output_31_1.png}
    \end{center}
    { \hspace*{\fill} \\}
    

    % Add a bibliography block to the postdoc
    
    
    
    \end{document}
